\section{Dodawanie}

Do realizacji dodawania z przeniesieniem została wykorzystana instrukcja \textit{adcl}, która poza umieszczeniem sumy operandów w docelowym rejestrze, zapisuje stan przeniesienia we fladze przeniesienia (CF). Gdy stan CF jest wysoki, operacja \textit{adcl} uwzględnia go w wyniku dodawania, co ułatwia propagację przeniesień w dodawaniu liczb dłuższych, niż pozwala na to pojedyncze słowo maszynowe (w omawianym rozwiązaniu 32-bit).

\begin{lstlisting}[language={[x86masm]Assembler}, caption={Pętla realizująca dodawanie.}]
loop_add:
    # Sciagam stan flag ze stosu
    popf
    # Wczytuje wartosci do rejestrow %eax, %ebx.
    movl    liczba1(, %edi, 4), %eax
    movl    liczba2(, %edi, 4), %ebx
    # Dodaje z uwzglednieniem przeniesienia
    adcl    %ebx, %eax
    # Odkladam sume czesciowa na stos.
    pushl   %eax
\end{lstlisting}

\subsection{W zakresie}


\begin{lstlisting}[language={[x86masm]Assembler}, caption={Pętla realizująca dodawanie.}]
    liczba1:
    .long 0x10304008, 0x701100FF, 0x45100020, 0x08570030
liczba_zapetlen = (. - liczba1) / 4 - 1

liczba2:
    .long 0xF040500C, 0x00220026, 0x321000CB, 0x04520031
    \end{lstlisting}
    

\begin{figure}[H]
    \includegraphics[width=\linewidth]{screenshots/gdb_dodawanie_w_zakresie.png}
\end{figure}

\subsection{Poza zakresem}

\begin{figure}[H]
    \includegraphics[width=\linewidth]{screenshots/gdb_dodawanie_overflow.png}
\end{figure}
  